\section{Demonstration of \deploy capabilities}
The \deploy capabilities will be demonstrated through numerical
experiments. 
The numerical experiments are in the form of fuel cycle scenarios 
where the demand driving commodity, its demand curve and the 
combination of facilities in the scenario are varied. 
Each numerical experiment will be run for each prediction
algorithm. 
The prediction algorithms will be compared to determine each 
of their strengths and weaknesses. 
And how their overall performance demonstrates \deploy's 
capabilities. 

The main constraint that \deploy is trying to meet is to 
minimize the number of time steps where demand exceeds supply or 
supply exceeds capacity. 
Therefore, the basis of comparison for the numerical experiments 
are the number of time steps where demand exceeds supply (under 
supply) or supply exceeds capacity (under capacity). 
The distinction between under supply and under capacity is 
depending on whether the facility that provides that commodity 
is controlled by the demand driven deployment institution or 
supply driven deployment institution. 

The numerical experiments are broken down into four types: 
front-end facility deployment, back-end facility deployment, 
closed fuel cycle and transition scenario. 

All the results and code used to produce these numerical 
experiments can be found in their respective Ipython
notebooks in the demonstrations directory of \cite{deploydoi_2019}. 

\subsection{Front-end Deployment}

\begin{table}[h]
	\centering
	\caption {Front-end Deployment Numerical Experiments}
	\label{tab:fenum}
	\begin{tabular}{|l|p{2.75cm}|p{2.5cm}|p{2.1cm}|l|}
		\hline
		\textbf{\shortstack{Test \\ Scenario}} & \textbf{\shortstack{Facilities \\ Present}} & \textbf{\shortstack{Reactor \\ Parameters}} & \textbf{\shortstack{Driving \\ Commodity}} & \textbf{\shortstack{Demand \\ Equation}}\\
		\hline
		1 & \texttt{Source}, \texttt{Sink} & - & Fuel & 1000t\\
		\hline
		2 & \texttt{Source}, \texttt{Reactor} & Cycle time: 1, Refuel time: 0 & Power & 1000t\\
		\hline
		3 & \texttt{Source}, \texttt{Reactor} & Cycle time: 3, Refuel time: 1 & Power & 1000t\\
		\hline
		4 & \texttt{Source}, \texttt{Reactor}, \texttt{Sink} & Cycle time: 1, Refuel time: 0 & Power & $10(2.5)^{t/12}$\\
		\hline
	\end{tabular}
\end{table}

\begin{table}[h]
	\centering
	\caption {Test scenario 1 under supply results for each algorithm and commodity}
	\label{tab:scenario1}
	\begin{tabular}{|l|l|}
		\hline
		\textbf{\shortstack{Prediction \\ Algorithm}} & \textbf{\shortstack{Fuel \\ Under Supply}}\\
		\hline
		ma & 0\\
		\hline
		arch & 0\\
		\hline
		arma & 0\\
		\hline
		exp smoothing & 0\\
		\hline
		fft &  32\\
		\hline
		holt winters & 0\\
		\hline
		poly & 21\\
		\hline
	\end{tabular}
\end{table}

Table \ref{tab:scenario1} shows that the ma, arch, arma, exp 
smoothing and holt winters methods perform best to minimize
 under supply of fuel during the simulation. 

\subsubsection{Test Scenario 2}

\begin{table}[h]
	\centering
	\caption {Test scenario 2 under supply results for each algorithm and commodity}
	\label{tab:scenario2}
	\begin{tabular}{|l|l|l|}
		\hline
		\textbf{\shortstack{Prediction \\ Algorithm}} & \textbf{\shortstack{Power \\ Under Supply}}& \textbf{\shortstack{Fuel \\ Under Supply}}\\
		\hline
		ma & 0 & 28\\
		\hline
		arch & 0 & 24\\
		\hline
		arma & 0 & 12\\
		\hline
		exp smoothing & 0 & 24\\
		\hline
		fft &  23 & 3\\
		\hline
		holt winters & 0 & 24\\
		\hline
		poly & 40 & 19\\
		\hline
	\end{tabular}
\end{table}

Table \ref{tab:scenario2} shows that the arch, arma, exp 
smoothing, holt winters and ma methods perform best to minimize
under supply of power during the simulation.
Whereas, the fft method performs best to minimize under supply of 
fuel during the simulation. 

\subsubsection{Test Scenario 3}


\begin{table}[h]
	\centering
	\caption {Test scenario 3 under supply results for each algorithm and commodity}
	\label{tab:scenario3}
	\begin{tabular}{|l|l|l|}
		\hline
		\textbf{\shortstack{Prediction \\ Algorithm}} & \textbf{\shortstack{Power \\ Under Supply}}& \textbf{\shortstack{Fuel \\ Under Supply}}\\
		\hline
		ma & 24 & 27\\
		\hline
		arch & 26 & 24\\
		\hline
		arma & 36 & 33\\
		\hline
		exp smoothing & 35 & 18\\
		\hline
		fft &  22 & 4\\
		\hline
		holt winters & 35 & 18\\
		\hline
		poly & 38 & 31\\
		\hline
	\end{tabular}
\end{table}

Table \ref{tab:scenario3} shows that the fft method performs best 
to minimize under supply of power and fuel during the simulation. 

\subsubsection{Test Scenario 4}

\begin{table}[h]
	\centering
	\caption {Test scenario 4 under supply results for each algorithm and commodity}
	\label{tab:scenario4}
	\begin{tabular}{|l|l|l|}
		\hline
		\textbf{\shortstack{Prediction \\ Algorithm}} & \textbf{\shortstack{Power \\ Under Supply}}& \textbf{\shortstack{Fuel \\ Under Supply}}\\
		\hline
		ma & 0 & 8\\
		\hline
		arch & 0 & 12\\
		\hline
		arma & 1 & 10\\
		\hline
		exp smoothing & 0 & 14\\
		\hline
		fft &  7 & 7\\
		\hline
		holt winters & 0 & 14\\
		\hline
		poly & 11 & 7\\
		\hline
	\end{tabular}
\end{table}

Table \ref{tab:scenario4} shows that the ma, arch, exp smoothing and holt winters
methods perform best to minimize
under supply of power during the simulation.
Whereas, the fft and poly method performs best to minimize under supply of 
fuel during the simulation. 

\subsection{Back-end Deployment}

\begin{table}[h]
	\centering
	\caption {Back-end Deployment Numerical Experiments}
	\label{tab:benum}
	\begin{tabular}{|l|p{2.75cm}|p{2.5cm}|p{2.1cm}|l|}
		\hline
		\textbf{\shortstack{Test \\ Scenario}} & \textbf{\shortstack{Facilities \\ Present}} & \textbf{\shortstack{Reactor \\ Parameters}} & \textbf{\shortstack{Driving \\ Commodity}} & \textbf{\shortstack{Demand \\ Equation}}\\
		\hline
		5 & \texttt{Source}, \texttt{Reactor}, \texttt{Sink} & Cycle time: 1, Refuel time: 0 & Power & 1000t\\
		\hline
		6 & \texttt{Source}, \texttt{Reactor}, \texttt{Storage}, \texttt{Sink} & Cycle time: 1, Refuel time: 0 & Power & 1000t\\
		\hline
	\end{tabular}
\end{table}

\subsubsection{Test Scenario 5}

\begin{table}[h]
	\centering
	\caption {Test scenario 5 under supply results for each algorithm and commodity}
	\label{tab:scenario5}
	\begin{tabular}{|l|l|l|l|}
		\hline
		\textbf{\shortstack{Prediction \\ Algorithm}} & \textbf{\shortstack{Power \\ Under Supply}}& \textbf{\shortstack{Fuel \\ Under Supply}} & \textbf{\shortstack{Spent Fuel \\ Under Capacity}}\\
		\hline
		ma & 0 & 28 & 7\\
		\hline
		arch & 0 & 24 & 7\\
		\hline
		arma & 0 & 12 & 7\\
		\hline
		exp smoothing & 0 & 24 & 7\\
		\hline
		fft &  23 & 3 & 8\\
		\hline
		holt winters & 0 & 24 & 7\\
		\hline
		poly & 40 & 19 & 7\\
		\hline
	\end{tabular}
\end{table}

Table \ref{tab:scenario5} shows that the ma, arch, arma, exp 
smoothing and holt winters methods perform best to minimize
under supply of power during the simulation.
The fft method performs best to minimize 
under supply of fuel during the simulation. 
The ma, arch, arma, exp smoothing, holt winters and poly methods 
perform best to minimize under supply of spent fuel in the 
simulation. 

\subsubsection{Test Scenario 6}

\begin{table}[h]
	\centering
	\caption {Test scenario 6 under supply results for each algorithm and commodity}
	\label{tab:scenario6}
	\begin{tabular}{|l|l|l|l|l|}
		\hline
		\textbf{\shortstack{Prediction \\ Algorithm}} & \textbf{\shortstack{Power \\ Under Supply}}& \textbf{\shortstack{Fuel \\ Under Supply}} & \textbf{\shortstack{Spent Fuel \\ Under Capacity}} & \textbf{\shortstack{Cool Spent Fuel \\ Under Capacity}}\\
		\hline
		ma & 0 & 28 & 3 & 11\\
		\hline
		arch & 0 & 24 & 3 & 14\\
		\hline
		arma & 0 & 12 & 3 & 14\\
		\hline
		exp smoothing & 0 & 24 & 3 & 11 \\
		\hline
		fft &  23 & 3 & 4 & 8\\
		\hline
		holt winters & 0 & 24 & 3 & 11\\
		\hline
		poly & 40 & 19 & 3 & 7\\
		\hline
	\end{tabular}
\end{table}


\subsection{Closed Fuel Cycle}

\begin{table}[h]
	\centering
	\caption {Closed Fuel Cycle Deployment Numerical Experiment}
	\label{tab:cfcnum}
	\begin{tabular}{|l|p{2.75cm}|p{2.5cm}|p{2.1cm}|l|}
		\hline
		\textbf{\shortstack{Test \\ Scenario}} & \textbf{\shortstack{Facilities \\ Present}} & \textbf{\shortstack{Reactor \\ Parameters}} & \textbf{\shortstack{Driving \\ Commodity}} & \textbf{\shortstack{Demand \\ Equation}}\\
		\hline
		7 & \texttt{Source}, \texttt{Reactors}, \texttt{Separation Facilities}, \texttt{Mixer Facilities} & Cycle time: 1, Refuel time: 0 & Power & 1000t\\
		\hline
	\end{tabular}
\end{table}

\subsection{Transition Scenario}
